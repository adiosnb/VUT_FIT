% !TeX spellcheck = sk_SK
\documentclass[a4paper, 11pt]{article}

\usepackage[czech]{babel}
\usepackage[utf8x]{inputenc}  % pro unicode UTF-8
\usepackage{amsmath}
\usepackage{amsthm}
\usepackage{graphics}
\usepackage{epstopdf}
\usepackage{multirow}
\usepackage{caption}
\usepackage{picture}
\usepackage[linesnumbered,longend,ruled]{algorithm2e}
\usepackage{rotating}
\usepackage{pdflscape}
\usepackage{listings}
%\usepackage[IL2]{fontenc}

\usepackage{geometry}
\geometry{
 a4paper,
 total={160mm,240mm},
 left=25mm,
 top=30mm,
}

\usepackage{times}


\begin{document}

\begin{center}
\pagenumbering{gobble}
\thispagestyle{empty}
\Huge
\textsc{Vysoké učení technické v Brně\\
\huge Fakulta informačních technologií\\}

\vspace{\stretch{0.382}}
\LARGE IMP -  Mikroprocesorové a vestavěné systémy\\
\Huge ARM/FITkit3: Zabezpečení dat pomocí 16/32-bit. kódu CRC, S 
\vspace{\stretch{0.618}}
\end{center}
{\LARGE \today \hfill Adrián Tomašov}
\pagebreak

\pagenumbering{roman}
%OBSAH
\tableofcontents
\pagebreak

\pagenumbering{arabic}
\setcounter{page}{1}


\section{Úvod}
Účelom tohto projektu bolo demonštrovať rôzne spôsoby zabezpečenia dát pomocou kódu 16/32-CRC. Tento projekt bol vyvíjaný na doske FitKit3 Minerva s čipom kinetis K60. Výpočty CRC sme mali robiť tromi rôznymi spôsobmi:
 \begin{enumerate}
	\item Použitím hardwarového modulu pre výpočet 16/32-bitového CRC z čipu K60
	\item Výpočet na základe pomocnej vyhľadávacej tabuľky, ktorá urýchli výpočty CRC
	\item Výpočet CRC pomocou polynómu
\end{enumerate}



\section{Vývoj}
Tento projekt bol vyvíjaný na virtuálnom stroji s operačným systémom Windows 7. Virtuálny stroj bol spustený na systéme Fedora 27 a vývojová doska bola premostená cez USB do virtuálneho stroja. Program pre virtuálizáciu použitý VirtualBox.

Takýto spôsob vývoja bol zvolený, pretože nastali isté komplikácie pri inštalácií a konfigurovacií vývojového prostredia Kinetis Design Studio na operačnom systéme Fedora 27. Pre komunikáciu užívateľa s vývojovou doskou bol použitý program Putty.



\section{Kostra programu}
Kostru tohto programu je demo program z Kinetis SDK(Software development kit) 2.0, kde sa nachádzali ukážky kódu pre používanie hardwarového modulu, ktorý počíta CRC. V tomto demo programe bol celý projekt prednastavený pre správny preklad a spustenie kódu na vývojovej doske, čo mi značne uľahčilo ďalší postup vo vývoji.

V tejto kostre som pri implementácii ostatných CRC algoritmov upravoval len jeden súbor a mal názov {\tt crc.c}.

\section{CRC}

Kontrola správnosti cyklickým kódom (Cyclic redundancy check) je spôsob zabezpečenia dát kontrolným súčtom. Tento súčet slúži pre odhalenie chyby v dátach, nad ktorými ju súčet vypočítaný. Najčastejšie sa tento súčet používa v telekomunikačnej technike a počítačových sieťach. Odosielateľ dát vypočíta CRC súčet a pri odosielaní pridá tento súčet k odoslaným dátam. Na strane prijímateľa sa tento proces opakuje a výsledná hodnota CRC súčtu sa porovná so súčtom, ktorý poslal odosielateľ. Ak sú súčty rovnaké, znamená to, že sa dáta preniesli bezchybne. 

\subsection{CRC v hardwarovom module}

V čipe K60 ja zabudovaný hardwarový modul pre akcelerované výpočty CRC. Pre uľahčenie práce s týmto modulom bolo použité \textbf{Kinetis SDK 2.0}, kde bol implementovaný driver pre ovládanie CRC modulu. Pred použitím modulu bolo nutné nakonfigurovať tento modul so správnymi parametrami daného CRC algoritmu. Táto konfigurácia sa ukladala do premennej {\tt crc\_config\_t config}. Hodnoty pre CRC algoritmy sa dali ľahko nájsť kdekoľvek na internete. Následne sa zavolala funkcia \linebreak {\tt CRC\_Init(base, \&config)}, ktorá nakonfigurovala daný CRC modul. Táto funckia sa volala niekoľko krát, vždy keď sa menil CRC algoritmus.

Pre zápis dát do modulu sa používala funkcia {\tt CRC\_WriteData(base, (uint8\_t *)data, size)}, ktorá zapísala do CRC registra dáta, z ktorých sa počítal CRC súčet. Pre výsledok súčtu je v \textbf{ Kinetis SDK 2.0} implementovaná funkcia {\tt  CRC\_Get16bitResult()} a {\tt CRC\_Get32bitResult()}.

\subsubsection{CRC16 CCINT-FALSE - konfigurácia}
\begin{lstlisting}
	115 static void InitCrc16_CcitFalse(CRC_Type *base, uint32_t seed)
	116 {   
	117     crc_config_t config;
	118     
	119     CRC_GetDefaultConfig(&config);
	120     config.seed = seed;
	121     CRC_Init(base, &config);
	122 }
	
\end{lstlisting}

\subsubsection{CRC32 - konfigurácia}
	\begin{lstlisting}
	125 static void InitCrc32(CRC_Type *base, uint32_t seed)
	126 {   
	127     crc_config_t config;
	128     
	129     config.polynomial = 0x04C11DB7U;
	130     config.seed = seed;
	131     config.reflectIn = true;
	132     config.reflectOut = true;
	133     config.complementChecksum = true;
	134     config.crcBits = kCrcBits32;
	135     config.crcResult = kCrcFinalChecksum;		136 
	137     CRC_Init(base, &config);
	138 }
	
	\end{lstlisting}





\subsection{CRC pomocou polynómu}
Toto je základný algoritmus pre výpočet CRC súčtu. Nemá v sebe žiadne akcelerácie ani optimalizácie pre výpočet. CRC hodnota je uložená v pomocnej premennej a upravuje sa počas celého hlavného cyklu v algoritme. Algoritmus cyklí cez každý bit. V každom kroku cyklu bitovo posunie doľava/doprava premennú s CRC súčtom. Ak aktuálny bit je nastavený na logickú 1, tak ešte s premennou CRC urobí binárnu operáciu XOR s polynómom. Po ukončení cyklu je v premennej CRC uložený správny súčet CRC.

\subsection{CRC pomocou vyhľadávacej tabuľky}
Toto je optimalizovaný algoritmus pre výpočet CRC. Pre urýchlenie výpočtov využíva vopred vygenerovanú tabuľku s pomocnými výpočtami. V tomto algoritme sa cyklí cez bajty a nie bity. V premennej CRC sa ukladá postupne vypočítaný súčet CRC. V každom kroku cyklu sa urobí operácia XOR aktuálneho bajtu a CRC bitovo posunutým v vľavo/vpravo. Výsledok tejto operácie je hodnota, ktorá slúži pre prístup k číslu z pomocnej tabuľky. Táto hodnota je pomocou operácie XOR pripočítaná k hodnote CRC, ktorá je bitovo posunutá vľavo/vpravo.

\section{Porovnanie}
Z týchto spôsobov je najrýchlejší hardwarový CRC modul. Nepotrebuje pre svoj chod procesor a svoje výpočty robí priamo na logických obvodoch. Výhody : nízka pamäťová náročnosť, rýchlosť.

Výpočet CRC pomocou matematický výpočtov je najpomalší, pretože cyklí cez každý bit v dátach. Nad týmto bitom sa ďalej robí porovnanie a jednu alebo dve matematické operácie nad premennou. Výhody: nízka pamäťová náročnosť. Nevýhody: tento algoritmus je pomalý.

CRC výpočet pomocou vyhľadávacej tabuľky je, v porovnaní s hardware CRC modulom a matematickým výpočtom, niekde medzi nimi. Cyklí cez všetky bajty a nie bity. Táto vlastnosť ho robí rýchlejším algoritmom oproti výpočtu CRC cez matematický polynóm. Výhody: rýchlosť. Nevýhody: veľké pamäťové nároky kvôli vyhľadávacej tabuľke.


\end{document}
