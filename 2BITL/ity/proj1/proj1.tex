\documentclass[a4paper, twocolumn, 11pt]{article}

\usepackage[czech]{babel}
\usepackage[utf8x]{inputenc}  % pro unicode UTF-8
\usepackage[IL2]{fontenc}

\usepackage{geometry}
\geometry{
 a4paper,
 total={170mm,240mm},
 left=20mm,
 top=25mm,
}

%\fontfamily{times}
\usepackage{times}

\begin{document}

\title{
Typografie a publikování\\
1. projekt}
\author{Adrián Tomašov\\
xtomas32@stud.fit.vutbr.cz}
\date{}
\maketitle


\section{Hladká sazba}
Hladká sazba je sazba z~jednoho stupně, druhu a řezu pí­sma sázená na stanovenou šířku odstavce. Skládá se z~odstavců, které obvykle začínají­ zarážkou, ale mohou být sázeny i bez zarážky - rozhodují­cí­ je celková grafická úprava. Odstavce jsou ukončeny východovou řádkou. Věty nesmějí začínat číslicí.

Barevné zvýraznění­, podtrhávání­ slov či různé velikosti písma vybraných slov se zde také nepoužívá. Hladká sazba je určena především pro delší­ texty, jako je napří­klad beletrie. Porušení­ konzistence sazby působí v~textu rušivě a~unavuje čtenářův zrak.

\section{Smíšená sazba}

Smíšená sazba má o~něco volnější­ pravidla než hladká sazba. Nejčastěji se klasická hladká sazba doplňuje o~další řezy pí­sma pro zvýraznění­ důležitých pojmů. Existuje \quotedblbase pravidlo":
\begin{quotation}
Čí­m ví­ce \textbf{druhů, \emph{řezů},}  {\scriptsize velikostí}, barev pí­sma a jiných efektů použijeme, tí­m \textit{profesionálněji} bude  dokument vypadat. Čtenář tím bude vždy {\Huge nadšen!}
\end{quotation}

\textsc{Tí­mto pravidlem se \underline{nikdy} nesmí­te ří­dit.}\linebreak Příliš časté zvýrazňování textových elementů  a změny velikosti {\tiny pí­sma} jsou {\LARGE známkou} \textbf{\huge amatéris\- mu} autora a působí­ \textbf{\textit {velmi}} rušivě. Dobře navržený dokument nemá obsahovat ví­ce než 4 řezy či druhy pí­sma. {\tt Dobře navržený dokument je decentní­, ne chaotický.}

Důležitým znakem správně vysázeného dokumentu je konzistentní použí­vání­ různých druhů zvýraznění­. To napří­klad může znamenat, že \textbf{ tučný řez} pí­sma bude vyhrazen pouze pro klíčová slova, \emph{skloněný řez} pouze pro doposud neznámé pojmy a nebude se to míchat. Skloněný řez nepůsobí­ tak rušivě a použí­vá se častěji. V~\LaTeX u \space jej sází­me raději pří­kazem\verb! \emph{text}! než\verb! \textit{text}!.

Smíšená sazba se nejčastěji používá pro sazbu vědeckých článků a technických zpráv. U~delší­ch dokumentů vědeckého či technického charakteru je zvykem upozornit čtenáře na význam různých typů zvýrazně\-ní­ v~úvodní­ kapitole.

\section{České odlišnosti}

Česká sazba se oproti okolní­mu světu v~některých as\-pektech mí­rně liší­. Jednou z~odlišností je sazba uvozovek. Uvozovky se v~češtině použí­vají­ převážně pro zobrazení­ pří­mé řeči. V~menší­ míře se použí­vají­ také pro zvýraznění­ přezdí­vek a ironie. V~češtině se použí\-­vá tento \textbf{\quotedblbase typ uvozovek\textquotedblleft} namí­sto anglických \textquotedblleft uvo\-zovek". Lze je sázet připravenými příkazy nebo při \linebreak použití UTF-8 kódování i přímo.

Ve smíšené sazbě se řez uvozovek ří­dí­ řezem první­ho uvozovaného slova. Pokud je uvozována celá věta, sází­ se koncová tečka před uvozovku, pokud se uvozuje slovo nebo část věty, sází­ se tečka za uvozovku.

Druhou odlišností je pravidlo pro sázení­ konců řád\-ků. V~české sazbě by řádek neměl končit osamocenou jednopí­smennou předložkou nebo spojkou. Spojkou \quotedblbase a\textquotedblleft \space končit může při sazbě do 25 liter. Abychom \LaTeX u zabránili v~sázení­ osamocených předložek, vkládáme mezi předložku a slovo \textbf{nezlomitelnou mezeru} znakem\verb! ~! (vlnka, tilda). Pro automatické do\-plnění vlnek slouží­ volně šiřitelný program \emph{vlna} od pana Olšáka\footnote{Viz http://petr.olsak.net/ftp/olsak/vlna/.}.


\end{document}