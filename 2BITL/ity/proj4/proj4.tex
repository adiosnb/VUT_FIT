\documentclass[a4paper, 11pt]{article}

\usepackage[czech]{babel}
\usepackage[utf8x]{inputenc}  % pro unicode UTF-8
\usepackage{amsmath}
\usepackage{amsthm}
\usepackage{graphics}
\usepackage{epstopdf}
\usepackage{multirow}
\usepackage{caption}
\usepackage{picture}
\usepackage[linesnumbered,longend,ruled]{algorithm2e}
\usepackage{rotating}
\usepackage{pdflscape}
%\usepackage[IL2]{fontenc}

\usepackage{geometry}
\geometry{
 a4paper,
 total={170mm,240mm},
 left=20mm,
 top=30mm,
}

\usepackage{times}


\begin{document}

\begin{center}
\thispagestyle{empty}
\Huge
\textsc{Vysoké učení technické v Brně\\
\huge Fakulta informačních technologií\\}

\vspace{\stretch{0.382}}
\LARGE Typografie a publikování - 4. projekt\\
\Huge Citace
\vspace{\stretch{0.618}}
\end{center}
{\LARGE \today \hfill Adrián Tomašov}
\pagebreak


\section{Typografia}

Záujmom typografie je problematika grafickej úpravy tlačených dokumentov s použitím vhodných rezov písma a usporiadania jednotlivých znakov a odsekov vo vhodnej forme. Dizajn písma, farby a zalamovanie sú tiež smery, ktoré rozoberá typografia. Niekedy je označované za druh grafickeho umenia.

Preklad slova typografia je podľa slovníka cudzích slov buď odbor zahrňajúci sadzbu a kníhtlač alebo výtvarné a technické riešenie tlačoviny. \cite{def}

V sučastnosti začina byť zanedbábaná, pretože je mnoho mladých a neskusených autorov diel, ktorý si neuvedomujú dlhý vývoj tejto vednej disciplíny. Namiesto toho aby sa učili z kníh alebo manuálov o typografií, učia sa z vlastných chýb. \cite{forisek}

Vznik typografie sa považuje približne rok 1440, kedy  Johannes Gutenberg   vynašiel kníhtlač s využitím pohyblivých znakov. Do tejto doby boli výtlačky značne nekvalitné, za čo mohli zlé postupy pri výrobe. \cite{polyg} Johanes Gutenberg vymyslel aj novy typ pisma zvané \textbf{blackletter} alebo inak nazývané \textbf{Gothic, Fraktur, OldEnglish}. Toto písmo je špecifické tým, že má výrazné rodzdiely medzi hrubými a tenkými čiarami, ktoré občas doplňali pätky. Inšpiráciou mu bolo ručné písmo. V tej dobe vnikalo mnoho podobných blackletter písiem po celej západnej Európe, preto je dnes na vyber z celkom prestrej ponuky. \cite{blackletter}


\section{Písmo} 
Písmo je vyjadrenie zvykových hlások pomocou písanej formy. Najčastejšie sa používa latinka, ale medzi často použivané patrí napríklad azbuka alebo arabské písmo. \cite{chempc}

Začiatky písma, ktoré používame dneska, sú z Egypta a Mezopotámie. Písmo sa vyvíjalo mnoho storočí, preto môžu byť označovane aj ako gotické, renesančné alebo moderné.  Po vynáleze kníhtlače, písma niesli pomenovanie po svojich autoroch.\cite{znaksad}

Dnes máme na výber množstvo druhov písma. Medzi najpopulárnejšie patria Sans, Monospaced, Script, Venetian a mnoho ďalších. Výber správneho písma je dôležité pre pozdvihnutie váznamu daného textu. Tak isto by mal výber spĺňať aj očakávania čitaleľa. \cite {destype}

Jedno z rozdelení písiem je na pätkové a bezpätkové. Pätky vnikli ako prirodzené ukončenie písania. Ich tvar sa menil v závislosti na materiály. Bezpätkové písmo vniklo zjednodušovaním jednotlivých ťahov v danom písmene. \cite{basics}

Voľba písma by mala byť záväzná pre celé písané dielo. Kombinavanie rôznych druhov písma bez dovôdne nezanechá dobrý pocit na čitateľovi. Vhodné využitie rôznych písiem je vhodne napríklad pri priamych citáciach alebo pri vkladaní zdrojových kódov nejakého programu. Pri zaujímavý kontraste sa dá dosiahnuť aj kombinovaním pätkového a bezpätkového písma. \cite{combfont} 

\section{Distribúcia typografických diel}
Od vynálezu kníhtlače do dnes sa v oblastí distribúcie diel mnoho zmenilo. Za posledných 15 rokov sa podial digitálnej komunikácie stale zvyšoval. \cite{tlac} Dnes sú sidtribucie diel hlavne pomocou pdf a epub dokumentov.


\pagebreak
\bibliographystyle{czplain}
\renewcommand{\refname}{Literatúra}
\bibliography{liter}

\end{document}
