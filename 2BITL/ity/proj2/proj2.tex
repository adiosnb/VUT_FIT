\documentclass[a4paper, 11pt]{article}

\usepackage[czech]{babel}
\usepackage[utf8x]{inputenc}  % pro unicode UTF-8
\usepackage{amsmath}
\usepackage{amsthm}

%\usepackage[IL2]{fontenc}

\usepackage{geometry}
\geometry{
 a4paper,
 total={180mm,250mm},
 left=15mm,
 top=25mm,
}

%\fontfamily{times}
\usepackage{times}

\theoremstyle{definition}
\newtheorem{definition}{Definice}[section]
\theoremstyle{theorem}
\newtheorem{algorithm}[definition]{Algoritmus}
\newtheorem{veta}{Věta}


\begin{document}

\begin{center}
\thispagestyle{empty}
\Huge
\textsc{Fakulta informačních technologií\\
Vysoké učení technické v Brně}\\
\vspace{\stretch{0.382}}
Typografie a publikování - 2. projekt\\
\vspace{\stretch{0.618}}
\end{center}
{\LARGE 2017 \hfill Adrián Tomašov}

\newpage
\setcounter{page}{1}
\twocolumn

\section*{Úvod}
V~této úloze si vyzkoušíme sazbu titulní strany, matematických vzorců, prostředí a dalších textových struktur obvyklých pro technicky zaměřené texty (například rovnice \ref{firstEquation} nebo definice \ref{firstDefinition} na straně \pageref{firstDefinition}).

Na titulní straně je využito sázení nadpisu podle op\-tického středu s~využitím zlatého řezu. Tento postup byl probírán na přednášce.


\section{Matematický text}

Nejprve se podíváme na sázení matematických symbolů a~výrazů v~plynulém textu. Pro množinu $V$ označuje\linebreak card$(V)$ kardinalitu $V$.
Pro množinu $V$ reprezentuje $V^*$ volný monoid generovaný množinou $V$  s~operací konkatenace.
Prvek identity ve volném monoidu $V^*$ značíme symbolem $\varepsilon$.
Nechť $V^+ = V^*-\{\varepsilon\}$. Algebraicky je tedy $V^+$ volná pologrupa generovaná množinou $V$ s~operací konkatenace.
Konečnou neprázdnou množinu $V$ nazvěme abeceda.
Pro $w \in V^*$ označuje $|w|$ délku řetězce $w$. Pro $W \subseteq V$ označuje occur$(w, W)$ počet výskytů symbolů z~ $W$ v~řetězci $w$ a sym$(w,i)$ určuje $i$-tý symbol řetězce $w$; například sym$(abcd,3)=c$.

Nyní zkusíme sazbu definic a vět s~využitím balíku {\tt amsthm}.

\begin{definition}
\label{firstDefinition}
{\em Bezkontextová gramatika} je čtveřice $G=(V,T,P,S)$, kde $V$ je totální abeceda,
$T \subseteq V$ je abeceda terminálů, $S \in (V-T)$ je startující symbol a $P$\linebreak je konečná množina pravidel
tvaru $q\colon A\rightarrow \alpha$, kde\linebreak $A \in (V-T)$, $\alpha \in V^*$ a $q$ je návěští tohoto pravidla. Nechť $N=V-T$ značí abecedu neterminálů.
Po\-kud $q\colon A \rightarrow \alpha \in P$ , $\gamma,\delta \in V^*$ , $G$ provádí derivační\linebreak krok z $\gamma A \delta$ do $\gamma \alpha \delta$ podle pravidla $q\colon A \rightarrow \alpha$, sym\--bolicky píšeme
$\gamma A \delta \Rightarrow \gamma \alpha \delta$ $[q\colon A \rightarrow \alpha ] $ nebo zjed\-nodušeně $\gamma A \delta \Rightarrow \gamma \alpha \delta$ . Standardním způsobem definu\-jeme $\Rightarrow^m$, kde $m\geq0$ . Dále definujeme tranzitivní uzávěr\linebreak $\Rightarrow^+$ a tranzitivně-reflexivní uzávěr $\Rightarrow^*$ .
\end{definition}

Algoritmus můžeme uvádět podobně jako definice textově, nebo využít pseudokódu vysázeného ve vhodném prostředí (například {\tt algorithm2e)}.

\begin{algorithm}

Algoritmus pro ověření bezkontextovosti gramatiky. Mějme gramatiku G = (N, T, P, S).

\begin{enumerate}
 \item Pro každé pravidlo $p \in P$ proveď test, zda $p$ na levé straně obsahuje právě jeden symbol z~$N$.
 \item Pokud všechna pravidla splňují podmínku z~kroku~$1$, tak je gramatika $G$ bezkontextová.

\end{enumerate}


\end{algorithm}

\begin{definition}
Jazyk definovaný gramatikou  $G$ definujeme jako $L(G)=\{w \in T^* | S \Rightarrow^* w \} $.
\end{definition}

\subsection{ Podsekce obsahující větu}

\begin{definition}
Nechť $L$ je libovolný jazyk. $L$ je bezkontextový jazyk, když a jen když $L=L(G)$, kde $G$ je libovolná bezkontextová gramatika.
\end{definition}

\begin{definition}
Množinu $\mathcal{L}_{CF}=\{L|L$ je bezkontextový jazyk$\}$ nazýváme třídou bezkontextových jazyků.
\end{definition}

\begin{veta}
\label{firstVeta}
Nechť $L_{abc} \{ a^nb^nc^n | n \geq 0 \}$  Platí, že  $L_{abc} \not\in \mathcal{L}_{CF}$.
\end{veta}

\begin{proof}
Důkaz se provede pomocí Pumping lemma pro bezkontextové jazyky, kdy ukážeme, že není možné, aby platilo, což bude implikovat pravdivost věty \ref{firstVeta} .
\end{proof}

\section{Rovnice a odkazy}

Složitější matematické formulace sázíme mimo plynulý text. Lze umístit několik výrazů na jeden řádek, ale pak je třeba tyto vhodně oddělit, například příkazem \verb!\quad!.

$$
\sqrt[x^2]{y_0^3}\quad{N} =\{0,1,2,\ldots\}\quad x^{y^y}\neq x^{yy}\quad z_{i_j}\not\equiv z_{ij}
$$

V~rovnici (\ref{firstEquation}) jsou využity tři typy závorek s~různou explicitně definovanou velikostí.

\begin{eqnarray}
\label{firstEquation}
\left\{\Big[\left(a+b\right)*c\Big]^d+1\right\}&	=&x\\
\lim_{x \rightarrow\infty}\frac{\sin^2x+\cos^2x}{4}&=&y \nonumber
\end{eqnarray}

V~této větě vidíme, jak vypadá implicitní vysázení limity $\lim_{n\rightarrow\infty}f(n)$ v~normálním odstavci textu. Podobně je to i s~dalšími symboly jako $\sum_{1}^{n}$ 
či $\bigcup_{A \in \mathcal{B}}$ . V~případě vzorce $\lim\limits_{x\rightarrow0} \frac{\sin x}{x} = 1$ jsme si vynutili méně úspornou\linebreak sazbu příkazem 
\verb!\limits! .

\begin{eqnarray}
\int_{a}^{b}f(x)\textup{d}x  &=& - \int_{b}^{a}f(x)\textup{d}x\\
\left(\sqrt[5]{x^4}\right)^\prime=\left(x^{\frac{4}{5}}\right)^\prime &=&\frac{4}{5}x^{-\frac{1}{5}}=\frac{4}{5\sqrt[5]{x}}\\
\overline{\overline{A\vee B}}&=&\overline{\overline{A}\wedge\overline{B}}
\end{eqnarray}

\section{ Matice}

Pro sázení matic se velmi často používá prostředí {\tt array} a závorky (\verb!\left!, \verb!\right!).\pagebreak
\begin{equation*}
\begin{pmatrix}
a+b & b-a\\
\widehat{\xi+\omega} & \hat{\pi}\\
\vec{a} & \overleftarrow{AC}\\
0 & \beta
\end{pmatrix}
\end{equation*}

\begin{equation*}
\textbf{A}=
\begin{Vmatrix}
a_{11} & a_{12} &\dots & a_{1n}\\
a_{21} & a_{22} &\dots & a_{2n}\\
\vdots & \vdots & \ddots & \vdots\\
a_{m1} & a_{m2} &\dots & a_{mn}
\end{Vmatrix}
\end{equation*}

\begin{equation*}
\begin{vmatrix}
t&u\\
v & w
\end{vmatrix}
=tw -uv
\end{equation*}

Prostředí {\tt array} lze úspěšně využít i jinde.

\begin{equation*}
\binom{n}{k} = \begin{cases}
\frac{n!}{k!(n-k)!} & \textup{pro }  0 \leq  k \geq n \\
0 & \text{pro }  k \le 0 \ \textup{nebo }  k \ge n
\end{cases}
\end{equation*}

\section{Závěrem}

V~případě, že budete potřebovat vyjádřit matematickou konstrukci nebo symbol a nebude se Vám dařit jej nalézt v~samotném LaTeXu, doporučuji prostudovat možnosti balíku maker AmSLaTeX.
Analogická poučka platí obec\-ně pro jakoukoli konstrukci v~TeXu.



\end{document}