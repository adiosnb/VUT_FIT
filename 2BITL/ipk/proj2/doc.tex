\documentclass[a4paper, 11pt]{article}

\usepackage[czech]{babel}
\usepackage[utf8x]{inputenc}  % pro unicode UTF-8
\usepackage{amsmath}
\usepackage{amsthm}
\usepackage{graphics}
\usepackage{epstopdf}
\usepackage{multirow}
\usepackage{caption}
\usepackage{picture}
\usepackage[linesnumbered,longend,ruled]{algorithm2e}
\usepackage{rotating}
\usepackage{pdflscape}
\usepackage{listings}
%\usepackage[IL2]{fontenc}

\usepackage{geometry}
\geometry{
 a4paper,
 total={160mm,240mm},
 left=25mm,
 top=30mm,
}

\usepackage{times}


\begin{document}

\begin{center}
\pagenumbering{gobble}
\thispagestyle{empty}
\Huge
\textsc{Vysoké učení technické v Brně\\
\huge Fakulta informačních technologií\\}

\vspace{\stretch{0.382}}
\LARGE IPK - Počítačové komunikace a sítě\\
\Huge ARP - scanner
\vspace{\stretch{0.618}}
\end{center}
{\LARGE \today \hfill Adrián Tomašov}
\pagebreak

\pagenumbering{roman}
%OBSAH
\tableofcontents
\pagebreak


\pagenumbering{arabic}
\setcounter{page}{1}


\section{Abstrakt}
Cieľom tohto projektu je vytvoriť aplikáciu na skenovanie siete komunikujúcu prostredníctvom BSD socketov. Aplikácia bude pomocou ARP protokolu schopná skenovať lokálnu sieť, objaviť MAC adresy existujúcich zariadení a určiť ich mapovanie na IPv4 a IPv6 adresy. 

Pomocou argumentov príkazového riadku bude možné zadať sieťové rozhranie, na ktorom má skenovanie prebehnúť a súbor, do ktorého bude zapísaný výstup programu v~predpísanom formáte.

\section{Mapovanie IP adries na MAC adresy}
Pri sieťovej komunikácií je nevyhnutné prekladať IP adresy, pomocou, ktorých sa komunikuje na sieťovej vrstve na hardvérové (MAC) adresy, využívané na linkovej vrstve. Na tento účel existuje viacero protokolov a metód používaných IP verzie 4 a 6.

\subsection{ARP}
Address Resolution Protocol (ARP) je využívaný IPv4 na mapovanie IP adries na MAC adresy, ktoré sa používajú na linkovej vrstve. Protokolu ARP je venované RFC 826 \cite{arp}. Správy typu ARP request sú zasielané na broadcastovú IP adresu. Zariadenie, ktoré vlastní požadovanú adresu, (prípadne zariadenie, ktoré je súži ako jeho proxy) pošle odpoveď formou správy ARP reply.

\subsection{Mapovanie adries v~IPv6}
Na rozdiel od IPv4 je v~IPv6 na preklad adries používaný ICMPv6 protokol konkrétne správa \emph{Neighbor Solicitation}, ktorá je opísaná v~RFC 4861 \cite{ns}. Funguje na princípe výmeny správ s~dotazom na MAC adresy susedných uzlov v~lokálnej sieti. 

IPv6 ponúka výrazne väčší adresný priestor ako je to u~IPv4. Najpoužívanejší prefix pre IPv6 siete je \emph{64}. To znamená $2^{64} = 18446744073709551616$ možných IPv6 adries v~sieti. Použitie podobnej metódy ako pri IPv4 by malo neakceptovateľnú časovú náročnosť. Existuje množstvo rôznych metód ako skenovať IPv6 sieť, ako je priblížené v~RFC 7707. \cite{net_rec} V~tomto projekte sú použité 2 metódy s~nižšou časovú náročnosťou.

\subsection{Multicast ping}
Základom tejto metódy je \emph{icmpv6 echo request} na adresu \emph{FF02::1}. Táto adresa predstavuje all nodes multicast, čiže správa sa rozpošle všetkým zariadeniam v sieti. \cite{multicast} Väčšina normálnych systémov na túto správu odpovie (okrem Windows) a z odpovedí získame IPv6 a MAC adresy zariadení.

\subsection{Well-known ports}  
Hlavnou myšlienkou tohto typu získavania informácií je predpoklad, že sa v sieti nachádzajú sieťové prvky, ktoré majú IPv6 adresu podľa využitia (napríklad http-server $\rightarrow$ fe80::80, default-gateway $\rightarrow$ fe80::1, a pod.).
Na tieto adresy sa pošle \emph{Neighbor Solicitation} správa. Na rozdiel od \emph{echo request}, na tento typ správy odpovedá každé zariadenie v sieti.



\section{Návrh a implementácia}


\subsection{IPv4}
Program v~prvom kroku zistí IPv4 adresu danej sieťovej karty a vypočíta adresný rozsah. Následne pošle ARP request na všetky IPv4 adresy v~danej sieti. Všetky potrebné dátové typy sú definované v~\emph{NetStruct.h}, pretože hlavička ARP protokolu je v~kerneli odstránená podmieneným prekladom.


\subsection{Trieda Scanner}
Základom celého programu je trieda Scanner, v ktorej sú implementované funkcie pre prístup k systé\-movým informáciam o sieťovej karte, a práca so socketom. Pre správnu funkciu objektu danej triedy je nutné nastaviť názov portu, ktorý je možno vidieť vo výstupe príkazu \emph{ip l}. 

\subsubsection{Socket}
Táto trieda používa RAW socket, pretože umožňuje priamy prístup k dátam na sieťovej karte bez špecifi\-kovaného protokolu na štvrtej vrstve OSI modelu. Táto vlastnosť je využívaná pri správach ARP, pretože ARP je protokol druhej vrstvy.

\begin{lstlisting}
    if((sock_des = socket(PF_PACKET, SOCK_RAW,htons(ETH_P_ALL))) < 0 ) {
        throw "error open socket";
    }
\end{lstlisting}

\subsubsection{Scanning}
Všetko metódy skenovania sú spúšťané paralelne pomocou \emph{std::thread}. Po spustení týcho threadov sa čaká konštantný čas.

\subsubsection{Listener}
Listener je metóda, ktorá prijíma správy zo socketu, ktoré prichádzajú na sieťovú kartu. Ak je prijatá správa typu \emph{ARP Reply} alebo \emph{icmpv6}, uloží IPv4/6 a MAC adresu do vnútornej databázy. Hlavné telo funkcie je vo \verb! while(1){} ! cykle, preto je táto metóda v samostatnom threade. Po uplynutí vopred stanoveného času sa na tento thread aplikuje metóda \emph{detach()} a pokračuje sa k ukončeniu programu.

\section{XML}
Formát výstupu je zapísaný v tvare XML do súbora, ktorého názov je zadaný na vstupe programu za prepínačom \emph{-f}.

Pre lepšiu kompatibilitu rôznych linuxových systémov nie je použitá žiadna knižnica, ktorá preklada dátové štruktúry do XML. Preklad vnútornej štruktúry do formátu XML zabezpečuje metóda \emph{xml\_encode()} z triedy \emph{Devices}. Táto trieda predstavuje databázu všetkých nájdených zariadení v sieti. Ukladá si IPv4, IPv6 a MAC adresu o danom zariadení.

\section{Demo}

\subsection{demo 1}
\begin{lstlisting}
isa2015@isa2015:~/fit/ipk/proj2$ sudo ./ipk-scanner -i eth0 -f ipv4.xml
isa2015@isa2015:~/fit/ipk/proj2$ cat ipv4.xml
<?xml version="1.0" encoding="UTF-8"?>
<devices>
  <host mac="5254.0012.3500">
    <ipv4>10.0.2.2</ipv4>
  </host>
  <host mac="0800.27f5.108c">
    <ipv4>10.0.2.3</ipv4>
  </host>
  <host mac="0800.27c7.bab4">
    <ipv4>10.0.2.4</ipv4>
  </host>
  <host mac="0800.27dd.b2ad">
    <ipv4>10.0.2.5</ipv4>
  </host>
  <host mac="0800.270d.7ef2">
    <ipv4>10.0.2.6</ipv4>
  </host>
  <host mac="0800.27cc.7698">
    <ipv4>10.0.2.15</ipv4>
  </host>
</devices>
isa2015@isa2015:~/fit/ipk/proj2$ 

\end{lstlisting}

\subsection{demo 2}
\begin{lstlisting}
root@isa2015:/home/isa2015/fit/ipk/proj2# ./ipk-scanner-with-ipv6 -i eth1 -f ipv6.xml
isa2015@isa2015:~/fit/ipk/proj2$ cat ipv6.xml
<?xml version="1.0" encoding="UTF-8"?>
<devices>
  <host mac="0a00.2700.0000">
    <ipv4>192.168.56.1</ipv4>
    <ipv6>fe80::800:27ff:fe00:0</ipv6>
  </host>
  <host mac="0800.270f.2fe4">
    <ipv4>192.168.56.102</ipv4>
    <ipv6>fe80::a00:27ff:fe0f:2fe4</ipv6>
  </host>
  <host mac="0800.275d.bad1">
    <ipv4>192.168.56.100</ipv4>
  </host>
  <host mac="0800.27a9.1bdb">
    <ipv6>fe80::a00:27ff:fea9:1bdb</ipv6>
  </host>
</devices>

\end{lstlisting}

\pagebreak
\bibliographystyle{czplain}
\bibliography{liter}


\end{document}
